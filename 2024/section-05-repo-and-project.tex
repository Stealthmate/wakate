\section{To モノレポ or not to モノレポ?}

\begin{frame}
    \frametitle{どういうときにモノレポが良いのか?}
    \small
    \begin{block}{大前提}
        \begin{enumerate}
            \item<2-> 関係あるもの同士を同じ個体にまとめるのは管理コストを下げる.
            \item<3-> 個体を分割することは管理コストを上げる.
            \item<4-> 現実的に管理可能な個体の規模には限度がある.
        \end{enumerate}
    \end{block}
    \onslide<4>{
        \Large\emphasize{=> 原則モノレポが良い.}
    }
\end{frame}

\begin{frame}
    \frametitle{じゃ,どういうときにモノレポを分けるべきか?}
    会社のコードをすべて同じレポに入れたらどうなる?
    \begin{itemize}
        \item<2-> 人間による制約
        \begin{itemize}
            \item<3-> PR の処理が速すぎて追いつかない
            \item<4-> 情報が多すぎて整理できない (認知負荷が高い)
        \end{itemize}
        \item<5-> 人間以外による制約
        \begin{itemize}
            \item<6-> レポが重すぎて開けない
            \item<7-> 局所的に作業するツールがない
            \item<8-> Git そのものが落ちる
        \end{itemize}
        \item<9-> 人工的な制約
        \begin{itemize}
            \item<10-> 法律,監査要件等
        \end{itemize}
    \end{itemize}
    \onslide<11>{
        \emphasize{現実の上限にぶち当たるから分けるべき.}
    }
\end{frame}

\begin{frame}
    \frametitle{どうやって分けるか?}
    例えばこういう状況を想像してみよう
    \begin{itemize}
        \item<2-> MV 株式会社が2個のサービスを提供する
        \item<3-> サービス \textcolor{red}{A} はチーム \textcolor{red}{A} がレポ \textcolor{red}{A} で開発している
        \item<4-> サービス \textcolor{green}{B} はチーム \textcolor{green}{B} がレポ \textcolor{green}{B} で開発している
        \item<5-> チーム \textcolor{red}{A} と \textcolor{green}{B} はお互いに独立 (名前も顔も GH も知らない)
        \item<6-> サービス \textcolor{red}{A} と \textcolor{green}{B} は同じインフラ上にデプロイされる
        \item<7-> チーム \textcolor{blue}{C} がインフラを管理している
    \end{itemize}
    \onslide<8>{
        \begin{block}{練習問題}
            インフラ関連のコードはどこに置くべきか?
            \begin{enumerate}
                \item レポ \textcolor{red}{A} と \textcolor{green}{B} それぞれに置く
                \item 新たにレポ \textcolor{blue}{C} を作ってそこに置く
            \end{enumerate}
        \end{block}
    }
\end{frame}

\begin{frame}
    \frametitle{どうやって分けるか?}
    \begin{itemize}
        \item<2-> インフラ関連のコードの所有者が \textcolor{red}{A}, \textcolor{green}{B}, \textcolor{blue}{C} 全員.
        \item<3-> それぞれのチーム目線では,それぞれのレポで管理するのが最適.
        \begin{itemize}
            \item<4-> (復習)
            \item<5->「関係あるもの同士を同じ個体にまとめるのは管理コストを下げる.」
            \item<6->「個体を分割することは管理コストを上げる.」
        \end{itemize}
    \end{itemize}
    \onslide<7>{
        \Large\emphasize{=> A と C, B と C の利益が排反するから正解がない.}
    }
\end{frame}

\begin{frame}
    \LARGE
    ご清聴ありがとうございました!
\end{frame}

\begin{frame}
    \LARGE
    \only<1>{なんか...悲しい...}
    \only<2>{このまま終わらせるのか...?}
\end{frame}

% モノレポ = 単一バージョンではない」をどこかで主張したかも?