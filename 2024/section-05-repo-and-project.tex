\section{To モノレポ or not to モノレポ?}

\begin{frame}
    \frametitle{どういうときにモノレポが良いのか?}
    \small
    \begin{block}{大前提}
        \begin{enumerate}
            \item<2-> 関係あるもの同士を同じ個体にまとめるのは管理コストを下げる.
            \item<3-> 個体を分割することは管理コストを上げる.
        \end{enumerate}
    \end{block}
    \onslide<4>{
        \Large\emphasize{=> 原則モノレポが良い.}
    }
\end{frame}

\begin{frame}
    \frametitle{どういうときにモノレポを分けるべきか?}
\end{frame}

% 関係あるものは,基本的にわけない方が良い.
% 分けるのは分けざるを得ないとき = ツールのパフォ or 仕様でそもそも物理的に一緒にできないとき(pip とか,閲覧権とか),もしくは非技術的な要因でそうせざるを得なくなったとき(監査とか,法律とか)

% 分けざるを得ないとき「以外」に分けたらこうなる: 結合したまま分ける(= be/fe/infra で分ける等)と,結局別れてない= overhead が増えるだけ
