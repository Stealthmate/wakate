\section{To モノレポ or not to モノレポ?}

\begin{frame}
    \frametitle{どういうときにモノレポが良いのか?}
    \small
    \begin{block}{大前提}
        \begin{enumerate}
            \item<2-> 関係あるもの同士を同じ個体にまとめるのは管理コストを下げる.
            \item<3-> 個体を分割することは管理コストを上げる.
            \item<4-> 現実的に管理可能な個体の規模には限度がある.
        \end{enumerate}
    \end{block}
    \onslide<5>{
        \Large\emphasize{=> 上限を超えない限りモノレポが良い.}
    }
\end{frame}

\begin{frame}
    \frametitle{じゃ,どういうときにモノレポを分けるべきか?}
    会社のコードをすべて同じレポに入れたらどうなる?
    \begin{itemize}
        \item<2-> 人間による制約
        \begin{itemize}
            \item<3-> PR の処理が速すぎて追いつかない
            \item<4-> 情報が多すぎて整理できない (認知負荷が高い)
        \end{itemize}
        \item<5-> 人間以外による制約
        \begin{itemize}
            \item<6-> レポが重すぎて開けない
            \item<7-> 局所的に作業するツールがない
            \item<8-> Git そのものが落ちる
        \end{itemize}
        \item<9-> 人工的な制約
        \begin{itemize}
            \item<10-> 法律,監査要件等
        \end{itemize}
    \end{itemize}
    \onslide<11>{
        \emphasize{現実の上限にぶち当たるから分けるべき.}
    }
\end{frame}

\begin{frame}
    \frametitle{どうやって分けるか?}
    例えばこういう状況を想像してみよう
    \begin{itemize}
        \item<2-> MV 株式会社が2個のサービスを提供する
        \item<3-> サービス \textcolor{red}{A} はチーム \textcolor{red}{A},サービス \textcolor{green}{B} はチーム \textcolor{green}{B} が開発している.
        \item<5-> チーム \textcolor{red}{A} と \textcolor{green}{B} はお互いに独立 (名前も顔も GH も知らない).
        \item<6-> サービス \textcolor{red}{A} のコードとサービス \textcolor{green}{B} のコードは同じモノレポ \textcolor{blue}{C} 内にある.
        \item<7-> サービス \textcolor{red}{A} もサービス \textcolor{green}{B} もそれぞれ front-end と back-end から成る.
    \end{itemize}
    \onslide<8>{
        \normalfont
        \begin{block}{練習問題}
            レポ \textcolor{blue}{C} が大きくなりすぎてもうモノレポのままじゃ無理そうで分割しなければならない.どのように分割するか?
        \end{block}
    }
\end{frame}
