\documentclass{beamer}

% Preamble
\usepackage{luatexja-fontspec}
\setmainjfont{Noto Sans CJK JP}


% Title
\title{モノレポのすゝめ}
\author{Valeri Haralanov}
\institute{無所属}
\date{2024-09}

\begin{document}

\frame{\titlepage}


\section{モノレポとは?}
\begin{frame}
\frametitle{モノレポとは? - 世間の見解}
\begin{itemize}
    % https://en.wikipedia.org/wiki/Monorepo
    \item a monorepo is a software-development strategy in which the code for a number of projects is stored in the same repository (wiki)
    % https://monorepo.tools/
    \item A monorepo is a single repository containing multiple distinct projects, with well-defined relationships.
    % https://nx.dev/concepts/decisions/why-monorepos
    \item A monorepo is a single git repository that holds the source code for multiple applications and libraries, along with the tooling for them.
\end{itemize}

\end{frame}

\begin{frame}
\frametitle{何が嬉しいの? - 世間の見解}

\begin{itemize}
    % https://en.wikipedia.org/wiki/Monorepo
    \item No overhead to create new projects
    \item Shared code and visibility
    % https://monorepo.tools/
    \item Atomic commits across projects
    % https://nx.dev/concepts/decisions/why-monorepos
    \item One version of everything
\end{itemize}
\end{frame}

\begin{frame}
    \frametitle{異議あり!論点を掘り下げてみよう!}
    \begin{itemize}
        \item No overhead to create new projects - 「PJ」 と 「レポ」を混同してない?
        \item Shared code and visibility - どうやってシェアするの?ライブラリ?直接参照?それ,モノレポじゃなくてもできるくない?
        \item Atomic commits across projects - 実は異議なし.それは本当にそう.だけど,誤解しちゃだめ(lockstep リリース)
        \item One version of everything - それ本当に良いことなのか?
    \end{itemize}
\end{frame}

\begin{frame}
    \frametitle{PJ と レポの違い}
\end{frame}

\begin{frame}
    \frametitle{コードの共有}
\end{frame}

\begin{frame}
    \frametitle{}
\end{frame}

\end{document}