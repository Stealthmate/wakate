\section{モノレポとは?}
\begin{frame}
\frametitle{モノレポとは? --- 定義}
    \only<2>{
        \netquote{
            A monorepo is a software-development strategy in which the code for a number of projects is stored in the same repository.
        }{http://en.wikipedia.org/w/index.php?title=Monorepo&oldid=1221537048}{Wikipedia}
    }
    \only<3>{
        \netquote{
            A monorepo is a single repository containing multiple distinct projects, with well-defined relationships.
        }{https://monorepo.tools/}{monorepo.tools}
    }
    \only<4>{
        \netquote{
            A monorepo is a single git repository that holds the source code for multiple applications and libraries, along with the tooling for them.
        }{https://nx.dev/concepts/decisions/why-monorepos}{nx.dev}
    }
\end{frame}

\begin{frame}
\frametitle{モノレポとは? --- 例えば (OSS)}
\begin{itemize}
    \item \href{https://github.com/babel/babel}{Babel}
    \item \href{https://github.com/facebook/react}{React}
    \item \href{https://github.com/yarnpkg/yarn}{Yarn}
    \item \href{https://github.com/NixOS/nixpkgs}{NixOS}
    \item \href{https://github.com/ProtonMail/WebClients}{ProtonMail}
\end{itemize}
\end{frame}

\begin{frame}
    \frametitle{モノレポとは? --- 例えば (我々の自作)}
    \begin{itemize}
        \item 我々のレポ
        % TODO: link to real repo
    \end{itemize}
\end{frame}
